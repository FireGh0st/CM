\documentclass[notitlepage]{math}
\usepackage{lipsum}
\usetikzlibrary{patterns,positioning, decorations, decorations.pathreplacing}
\usepackage{newunicodechar}

\title{Numerical Series Chap. 1} %Titre du fichie
\author{FireGhost} %Auteur du fichier


\newcommand\Warning{%
 \makebox[1.4em][c]{%
 \makebox[0pt][c]{\raisebox{.1em}{\small!}}%
 \makebox[0pt][c]{\color{red}\Large$\bigtriangleup$}}}%

\newunicodechar{⚠}{\Warning}

\begin{document}
\titre{Chapter 1: Numerical Series} %Titre du fichier .pdf
\UE{Numerical Series} %Nom de la UE

\fairetitre
\fairemarges
% subsubsubsection
\setcounter{secnumdepth}{4}
\titleformat{\paragraph}
{\normalfont\normalsize\bfseries}{\theparagraph}{1em}{}
\titlespacing*{\paragraph}
{0pt}{3.25ex plus 1ex minus .2ex}{1.5ex plus .2ex}



\newcommand{\minus}{\scalebox{0.75}[1.0]{$-$}} % Minus sign

\section{Vocabulary}
In this chapter, we will use CVG for Convergence and DVG for Divergence. We will also use GT for General Term.

\section{General approach Convergence and Divergence}
\subsection{Definition}
Let $\bf(U_n)_{n \in \mathbb{N}}$ a sequence of real numbers, we call series of general term $\bf U_k$ and denote $\bf \sum U_k$ 
the sequence of partial sums $ \bf (S_n)_{n \in \mathbb{N}}$ where for any integer $\bf n \in \mathbb{N}$, $ \bf S_n = {\sum\limits_{k=0}^n U_k}$.
We say $\bf \sum U_k$ is convergent if and only if $\bf (S_n)_{n \in \mathbb{N}}$ is convergent.
\subsubsection{Example: the geometric series}
Let $\bf q \in \mathbb{R}^\star$ and let us consider the series $\bf \sum q^k$. We have:\\
\\
    $\forall n \in \mathbb{N}, S_n = \sum\limits_{k=0}^n q^k = \begin{array}{|rl}
        \frac{1 - q^{n+1}}{1 - q} & \text{if } q \neq 1  \implies
        \begin{array}{|r}
            \text{if } \minus 1 < q < 1, \sum\limits_{k=0}^{+\infty} q^k = \frac{1}{1-q} \sum U_k \text{: CVG} \\
            \text{if } q > 1 \text{ or } q < \minus 1, \sum U_k \text{: DVG} 
        \end{array} \\
        (n + 1) & \text{if } q = 1 \implies \sum U_k \text{: DVG}
    \end{array}$

\subsection{Propositions}
Let $\bf \sum U_k$ and $\bf \sum V_k$ two series of general terms and $\bf \lambda \in \mathbb{R}$. We have:
\begin{itemize}
    \item If [$\bf \sum U_k$ CVG and $\bf \sum V_k$ CVG], then $\bf \sum (U_k + V_k)$ CVG
    \item If [$\bf \sum U_k$ CVG], then $\bf \sum \lambda U_k$ CVG
    \item If [$\bf \sum U_k$ CVG and $\bf \sum V_k$ DVG], then $\bf \sum (U_k + V_k)$ DVG
    \item ⚠ $\bf \sum U_k$ DVG and $\bf \sum V_k$ DVG \underline{does not imply} $\bf \sum (U_k + V_k)$ DVG
\end{itemize}

\subsection{Sum and Remainder of a convergent series}
Let $\bf \sum U_k$ a \underline{convergent series}. We call sum of the series $\bf \sum U_k$ 
the following real number: $\bf \sum\limits_{k=0}^{+\infty} U_k = \lim\limits_{n \to +\infty} S_n$ where $\bf {S_n} = \sum\limits_{k=0}^n U_k$.
And we call remainder of the series $\bf \sum U_k$ sequence $\bf (R_n)$ defined as follows:
\[ \forall n \in \mathbb{N}, R_n = \sum\limits_{k=n+1}^{+\infty}\]
\subsubsection{Example}
$\bf \sum q^k$ CVG $\Leftrightarrow \minus 1 < q < 1$ : $\bf S = \lim\limits_{n \to +\infty} S_n = \frac{1}{1 - q}$

\subsection{Convergence necessary condition}
\subsubsection{Proposition}
Let $\bf \sum (U_k)_{k \in \mathbb{N}}$ a sequence. We have:
\[ \sum U_k \text{ CVG } \begin{array}{ r }\implies \\ \not\Longleftarrow \end{array} \left( U_k \xrightarrow[k \to +\infty]{} 0 \right) \]
\subsubsection{Example}
\begin{itemize}
    \item Harmonic series: $\bf \sum \frac{1}{n}$, $\bf (\frac{1}{n}) \xrightarrow[n \to +\infty]{} 0$ but $\bf \sum \frac{1}{n}$ DVG
    \item $\bf \sum \frac{e^n}{n^{2023}}$, $\frac{e^n}{n^{2023}} \xrightarrow[n \to +\infty]{} +\infty \implies \sum \frac{e^n}{n^{2023}}$ DVG
    \end{itemize}

\subsection{Positive Term Series (P.T.S.)}
\subsubsection{Definition}
Let $\bf \sum U_k$ a series. We say $\bf \sum U_k$ is a P.T.S., 
if and only if $\bf \forall k \in \mathbb{N}, U_k \geq 0$.\\
We say $\bf \sum U_k$ is a P.T.S. from $\bf p \in \mathbb{N}$ onwards,
if and only if $\bf \forall k \in \mathbb{N}, k \geq p \implies U_k \geq 0$.
\subsubsection{Propositions}
\begin{itemize}
    \item Let $\bf \sum U_k$ a P.T.S. and $\bf (S_n)_{n \in \mathbb{N}}$ the associated partial sum sequence. Then:
    \[ \sum U_k \text{ CVG } \Leftrightarrow (S_n)_{n \in \mathbb{N}} \text{ is upper-bounded} \]
    \item Let $\bf \sum U_k$ and $\bf \sum V_k$ two series such that: \\
    $\bf \forall k \in \mathbb{N}, 0 \leq U_k \leq V_k$. Then:
    \begin{enumerate}
        \item If $\bf \sum V_k$ CVG, then $\bf \sum U_k$ CVG
        \item If $\bf \sum U_k$ DVG, then $\bf \sum V_k$ DVG
    \end{enumerate}
\end{itemize}
\paragraph{Example}
What's the nature of $\bf \sum \frac{1}{\lvert n \cdot \sin(n) \rvert}$ ?\\

\noindent $\forall n \in \mathbb{N}^\star, 0 < \lvert \sin(n) \rvert \leq 1 \implies 0 < \frac{1}{n} \leq \frac{1}{\lvert n \cdot \sin(n) \rvert}$\\
$\bf \sum \frac{1}{n}$ (Harmonic) DVG $\implies \bf \sum \frac{1}{\lvert n \cdot \sin(n) \rvert}$ DVG

\section{Riemann's series}
\subsection{Definition}
We call Riemann's series any series of General Terms (GT) $\bf \sum \frac{1}{n^\alpha}$ where $\bf \alpha \in \mathbb{R}$.\\
\subsection{Theorem (Riemann)}
Let $\bf \alpha \in \mathbb{R}$. Then:
\[ \sum \frac{1}{n^\alpha} \text{ CVG } \Longleftrightarrow \alpha > 1 \]
\subsubsection{Example}
\begin{itemize}
    \item $\bf \sum \frac{1}{\sqrt{2}} = \sum \frac{1}{2^{\frac{1}{2}}} \implies$ DVG 
    \item $\bf \sum \frac{1 + cos(n)}{n^4}$: $\forall n \in \mathbb{N}^\star, 0 \leq 1 + cos(n) \leq 2 \implies 0 \leq \frac{1 + cos(n)}{n^4} \leq \frac{2}{n^4}$\\
    And $\bf \sum \frac{2}{n^4}$ of same nature as $\bf \sum \frac{1}{n^4}$ (Riemann's series) CVG $\implies \bf \sum \frac{1 + cos(n)}{n^4}$ CVG
\end{itemize}
\end{document}