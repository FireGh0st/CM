\documentclass[notitlepage]{math}
\usepackage{lipsum}
\usetikzlibrary{patterns,positioning, decorations, decorations.pathreplacing}
\title{Vector Spaces Chap. 11} %Titre du fichie
\author{matthiru} %Auteur du fichier

\begin{document}
\titre{Chapter 11: Vector Spaces} %Titre du fichier .pdf
\UE{Vector Spaces} %Nom de la UE

\fairetitre
\fairemarges
% subsubsubsection
\setcounter{secnumdepth}{4}

\titleformat{\paragraph}
{\normalfont\normalsize\bfseries}{\theparagraph}{1em}{}
\titlespacing*{\paragraph}
{0pt}{3.25ex plus 1ex minus .2ex}{1.5ex plus .2ex}


\section{Introduction}
\subsection{General approach}
Studying Vector spaces will allow us to notice general theorems that can be applied to many mathematical structures.
\subsection{Definition}
A vector space is a set whose elements, often called vectors, may be added together and multiplied ("scaled") by numbers called scalars.
\subsection{Notation}
A $\mathbb{K}-Vector\space Space$ is non-empty set $E$ that has :
\begin{itemize}
    \item An internal law, which is an map of $E$ x $E$ in $E$: \\
    \begin{tikzpicture}[node distance=1mm]
        \node (center) at (0,0) {};
        \node (functionName) at (4, -1) {};
        \node[right = of functionName] (domain) {$E \times E$};
        \node[right = 1cm of domain] (codomain) {$E$};
        \node[below = 2mm of domain] (element) {$(u , v)$};
        \node at (element-|codomain) (image) {$u+v$};
        \draw[->] (domain) -- (codomain);
        \draw[|->] (element) -- (image);
    \end{tikzpicture}
    \item An external law, which is an map of $\mathbb{K}$ x $E$ in $E$: \\
    \begin{tikzpicture}[node distance=1mm]
        \node (center) at (0,0) {};
        \node (functionName) at (4, -1) {};
        \node[right = of functionName] (domain) {$\mathbb{K} \times E$};
        \node[right = 1cm of domain] (codomain) {$E$};
        \node[below = 2mm of domain] (element) {$(\lambda , u)$};
        \node at (element-|codomain) (image) {$\lambda \cdot u$};
        \draw[->] (domain) -- (codomain);
        \draw[|->] (element) -- (image);
    \end{tikzpicture}
\end{itemize}
($\mathbb{K}$ is a set, often $\mathbb{R}$)\\ \\
The elements of $E$ are called \underline{vectors}\\
The elements of $\mathbb{K}$ are called \underline{scalars}\\
The neutral element $0_E$ is also called the null vector\\
The symmetrical $-u$  is also called the opposite\\
The internal composition law on $E$, denoted $+$, is the addition\\
The external composition law on $E$ is the multiplication by a scalar\\ \\
axioms relative to the internal law : 
\begin{itemize}
    \item $0_E$ is unique
    \item $-u$ is unique
\end{itemize}
\clearpage
\subsection{Properties}
To know is a space is a vector space, there is properties that need to match up.\\
\begin{itemize}
    \item The internal law
    \item The external law
    \item Both laws together
    \item The neutral element
    \item It's symmetrical
\end{itemize}
\underline{This makes up 8 laws that need to be respected :}
\begin{enumerate}
    \item $u + v = v + u \space\space\space(\forall (u,v) \in E)$
    \item $u + (v + w) = (u + v) + w \space\space\space(\forall (u,v,w) \in E)$  
    \item There exists a neutral element $0_E \in E$ so that $u + 0_E = u \space\space\space(\forall u \in E)$
    \item  All elements admit a symmetric $u'$ so that $u + u' = 0_E$. This element $u'$  is denoted $-u$ 
    \item $1 \cdot u  = u \space\space\space(\forall u\in E)$
    \item $\lambda \cdot (\mu \cdot u) = (\lambda \mu) \cdot u \space\space\space(\forall \space\lambda,\mu \in \mathbb{K} ,u \in E)$
    \item $\lambda \cdot (v + u) = (\lambda \cdot v)  + (\lambda \cdot u) \space\space\space(\forall \space\lambda \in \mathbb{K} ,v,u \in E)$
    \item $(\lambda + \mu) \cdot u = (\lambda \cdot u)  + (\mu \cdot u) \space\space\space(\forall \space\lambda,\mu \in \mathbb{K} ,u \in E)$
\end{enumerate}

\section{Vector Sub-Spaces}
A sub space is very useful to prove that a set is a Vector space. We will see that a vector sub-space is vector space.
\subsection{Definition}
Let $E$ be a vector space,  $F$ is a subspace if and only if
\begin{itemize}
    \item $0_E \in F$
    \item $u + v \in F$ $ \forall (u,v) \in F^2$
    \item $\lambda \cdot u \in F$ $ \forall \lambda \in \mathbb{K},v \in F$
\end{itemize}
\subsection{Properties}
Showing that a space is a subspace of a bigger (or equal) vector space if enough to prove that it is itself a vector space.
\section{Relations between V. Space and V. SubSpace}
\subsection{Linear Combinations}
Let $v_1,v_2,...,v_n$, $n$ vectors from a vector space $E$. Then :\\
Any vector of the form :
\[u = \lambda_1 v_1 + \lambda_2 v_2 + ... + \lambda_n v_n\]
is called a \underline{Linear Combination} of the vectors $v_1, v_2, ..., v_n$\\ \\
The scalars $\lambda_1, \lambda_2, ..., \lambda_n \in \mathbb{K}$ are the coefficients of the linear combination.\\\\
Let $E$ be a vector space. Then $F$ is a vector sub space if and only if all linear combination of two element of $F$ also belongs to $E$.

\subsection{Vector SubSpace Intersection}
\begin{itemize}
    \item the intersection $\cap$ of two vector sub spaces \textbf{is} also a sub space
    \item the union $\cup$ of two vector sub spaces \textbf{is not} a sub space
\end{itemize}
\subsection{Vector SubSpace Sum}
Let $F$ and $G$ be two vectorial sub spaces of $E$\\
The sum of two vectorial sub spaces is also a vss, in fact, it is the smallest vss including both  $F$ and $G$ \\
$F$ and $G$ are in direct sum in $E$ if
\begin{itemize}
    \item $F \cap G = \{0_E\}$
    \item $F + G = E$
\end{itemize}
We then denote $F \oplus G = E$\\
$F$  and $G$ are called additional sub-spaces\\\\
$F$ and $G$ are additional sub-spaces in $E$ if and only if any element of $E$ is uniquely written as the sum of an element of $F$ and an element of $G$ \\ \\
if $\begin{cases}
    w = u + v \\
    w = u' + v'
\end{cases}$
with $\begin{cases}
    u \in F, v \in G \\
    u' \in F, v' \in G
\end{cases}$
then $\begin{cases}
    u = u' \\
    v = v'
\end{cases}$


\section{Families}
\subsection{Free Family}
A family $\{v_1,v_2,...,v_p\}$  of $E$ is called a *free family*  (or *linearly independent family*) if all null linear combination\\
\[\lambda_1v_1 + \lambda_2v_2 + ... + \lambda_pv_p = 0\]\\
is such that all $\lambda$ coefficients are null.

The opposite (there exists a null linear combination with at least 1 coefficient that is not null) is called a \underline{linked family} or a \underline{linearly dependent family.}\\

Let $E$ a $\mathbb{K}$-vector space,\\
A family $F = \{v_1,v_2,...,v_p\}$  with $p \geq 2$ vectors of $E$ is a linked family iff at least one vector is a linear combination of the other vectors

\subsection{Spanning Family}
A family $\{v_1,v_2,...,v_p\} $ is a generative family of $E$ if all vector of $E$ is a linear combination of the vectors $v_1,v_2,...,v_p$ \\
\[\forall x \in E, \exists (\lambda_1,\lambda_2,...,\lambda_n) \in \mathbb{K}^n: x =\lambda_1x_1 + \lambda_2x_2 + ... + + \lambda_nx_n\]\\
To prove that a family spans a vector space, we need to find the solutions $\lambda_1,...,\lambda_n$ of the equation given just above.

\section{Basis}
A familly of vectors of $E$ is a basis if it is both a free family and a spanning family.\\
It acts as a reference for the vector space\\\\
Every basis of a finite dimension vector space $E$ as the same number of elements. This number is called the \textbf{dimension} (dim $E$) of the vector space.\\
TO COMPLETE !
\section{Dimension of a Vector SubSpace}
Let $E$ be a $\mathbb{K}$-vector space of dimension $n$. Then :
\begin{itemize}
    \item Any free family of $E$ as at most $n$ elements.
    \item Any spanning family of $E$ as at least $n$ elements.
\end{itemize}
Let $E$ be a $\mathbb{K}$-vector space of dimension $n$. Then :
\begin{itemize}
    \item A v.s.s of dimension $1$ is called a \colorbox{pink}{vector line}
    \item A v.s.s of dimension $2$ is called a \colorbox{pink}{vector plane}
    \item A v.s.s of dimension $n-1$ is called an \colorbox{pink}{hyperplan}
\end{itemize}
Let $F$ and $G$ be two v.s.s of $E$, a finite vector space.
\begin{itemize}
    \item $F$ is of a finite dimension too
    \item $\text{dim}\space F \leq \text{dim}\space E$
    \item $\text{dim}\space F = \text{dim}\space E \iff F = E$
\end{itemize}

\[\text{dim}(F + G)  = \text{dim}(F) + \text{dim}(G) - \text{dim}(F \cup G)\]
\section{Proofs}
\subsection{Intersection of linear subspaces}
\subsection{Sum of linear subspaces}
\subsection{Spanned linear subspaces}
\subsection{Basis}
\subsection{Existence of a basis in a finite dimention}

\end{document}