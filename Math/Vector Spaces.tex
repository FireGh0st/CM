\documentclass[notitlepage]{math}
\usepackage{lipsum}
\usetikzlibrary{patterns,positioning, decorations, decorations.pathreplacing}
\title{Vector Spaces Chap. 11} %Titre du fichie
\author{matthiru} %Auteur du fichier

\begin{document}
\titre{Chapter 11: Vector Spaces} %Titre du fichier .pdf
\UE{Vector Spaces} %Nom de la UE

\fairetitre
\fairemarges
% subsubsubsection
\setcounter{secnumdepth}{4}

\titleformat{\paragraph}
{\normalfont\normalsize\bfseries}{\theparagraph}{1em}{}
\titlespacing*{\paragraph}
{0pt}{3.25ex plus 1ex minus .2ex}{1.5ex plus .2ex}


\section{Introduction}
\subsection{General approach}
Studying Vector spaces will allow us to notice general theorems that can be applied to many mathematical structures.
\subsection{Definition}
A vector space is a set whose elements, often called vectors, may be added together and multiplied ("scaled") by numbers called scalars.
\subsection{Notation}
A $\mathbb{K}-Vector\space Space$ is non-empty set $E$ that has :
\begin{itemize}
    \item An internal law, which is an map of $E$ x $E$ : \\
    \begin{tikzpicture}[node distance=1mm]
        \node (center) at (0,0) {};
        \node (functionName) at (4, -1) {};
        \node[right = of functionName] (domain) {$E \times E$};
        \node[right = 1cm of domain] (codomain) {$E$};
        \node[below = 2mm of domain] (element) {$(u , v)$};
        \node at (element-|codomain) (image) {$u+v$};
        \draw[->] (domain) -- (codomain);
        \draw[|->] (element) -- (image);
    \end{tikzpicture}
\end{itemize}
\subsection{Properties}
\section{Vector Sub-Spaces}
\subsection{Definition}
\subsection{Properties}
\section{Relations between Vs and Vss}
\subsection{Linear Combinations}
\subsection{Vss Intersection}
\subsection{Vss Sum}
\subsection{Generated Vss}
\section{Families}
\subsection{Free Family}
\subsection{Spanning Family}
\section{Basis}
\section{Dimension of a Vss}
\section{Proofs}
\subsection{Intersection of linear subspaces}
\subsection{Sum of linear subspaces}
\subsection{Spanned linear subspaces}
\subsection{Basis}
\subsection{Existence of a basis in a finite dimention}

\end{document}