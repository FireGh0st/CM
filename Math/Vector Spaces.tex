\documentclass[notitlepage]{math}
\usepackage{lipsum}
\usetikzlibrary{patterns,positioning, decorations, decorations.pathreplacing}
\title{Vector Spaces Chap. 11} %Titre du fichie
\author{matthiru} %Auteur du fichier

\begin{document}
\titre{Chapter 11: Vector Spaces} %Titre du fichier .pdf
\UE{Vector Spaces} %Nom de la UE

\fairetitre
\fairemarges
% subsubsubsection
\setcounter{secnumdepth}{4}

\titleformat{\paragraph}
{\normalfont\normalsize\bfseries}{\theparagraph}{1em}{}
\titlespacing*{\paragraph}
{0pt}{3.25ex plus 1ex minus .2ex}{1.5ex plus .2ex}


\section{Introduction}
\subsection{General approach}
Studying Vector spaces will allow us to notice general theorems that can be applied to many mathematical structures.
\subsection{Definition}
A vector space is a set whose elements, often called vectors, may be added together and multiplied ("scaled") by numbers called scalars.
\subsection{Notation}
A $\mathbb{K}-Vector\space Space$ is non-empty set $E$ that has :
\begin{itemize}
    \item An internal law, which is an map of $E$ x $E$ in $E$: \\
    \begin{tikzpicture}[node distance=1mm]
        \node (center) at (0,0) {};
        \node (functionName) at (4, -1) {};
        \node[right = of functionName] (domain) {$E \times E$};
        \node[right = 1cm of domain] (codomain) {$E$};
        \node[below = 2mm of domain] (element) {$(u , v)$};
        \node at (element-|codomain) (image) {$u+v$};
        \draw[->] (domain) -- (codomain);
        \draw[|->] (element) -- (image);
    \end{tikzpicture}
    \item An external law, which is an map of $\mathbb{K}$ x $E$ in $E$: \\
    \begin{tikzpicture}[node distance=1mm]
        \node (center) at (0,0) {};
        \node (functionName) at (4, -1) {};
        \node[right = of functionName] (domain) {$\mathbb{K} \times E$};
        \node[right = 1cm of domain] (codomain) {$E$};
        \node[below = 2mm of domain] (element) {$(\lambda , u)$};
        \node at (element-|codomain) (image) {$\lambda \cdot u$};
        \draw[->] (domain) -- (codomain);
        \draw[|->] (element) -- (image);
    \end{tikzpicture}
\end{itemize}
($\mathbb{K}$ is a set, often $\mathbb{R}$)\\ \\
The elements of $E$ are called \underline{vectors}\\
The elements of $\mathbb{K}$ are called \underline{scalars}\\
The neutral element $0_E$ is also called the null vector\\
The symmetrical $-u$  is also called the opposite\\
The internal composition law on $E$, denoted $+$, is the addition\\
The external composition law on $E$ is the multiplication by a scalar\\ \\
axioms relative to the internal law : 
\begin{itemize}
    \item $0_E$ is unique
    \item $-u$ is unique
\end{itemize}
\clearpage
\subsection{Properties}
To know is a space is a vector space, there is properties that need to match up.\\
\begin{itemize}
    \item The internal law
    \item The external law
    \item Both laws together
    \item The neutral element
    \item It's symmetrical
\end{itemize}
\underline{This makes up 8 laws that need to be respected :}
\begin{enumerate}
    \item $u + v = v + u \space\space\space(\forall (u,v) \in E)$
    \item $u + (v + w) = (u + v) + w \space\space\space(\forall (u,v,w) \in E)$  
    \item There exists a neutral element $0_E \in E$ so that $u + 0_E = u \space\space\space(\forall u \in E)$
    \item  All elements admit a symmetric $u'$ so that $u + u' = 0_E$. This element $u'$  is denoted $-u$ 
    \item $1 \cdot u  = u \space\space\space(\forall u\in E)$
    \item $\lambda \cdot (\mu \cdot u) = (\lambda \mu) \cdot u \space\space\space(\forall \space\lambda,\mu \in \mathbb{K} ,u \in E)$
    \item $\lambda \cdot (v + u) = (\lambda \cdot v)  + (\lambda \cdot u) \space\space\space(\forall \space\lambda \in \mathbb{K} ,v,u \in E)$
    \item $(\lambda + \mu) \cdot u = (\lambda \cdot u)  + (\mu \cdot u) \space\space\space(\forall \space\lambda,\mu \in \mathbb{K} ,u \in E)$
\end{enumerate}

\section{Vector Sub-Spaces}
A sub space is very useful to prove that a set is a Vector space. We will see that a vector sub-space is vector space.
\subsection{Definition}
Let $E$ be a vector space,  $F$ is a subspace if and only if
\begin{itemize}
    \item $0_E \in F$
    \item $u + v \in F$ $ \forall (u,v) \in F^2$
    \item $\lambda \cdot u \in F$ $ \forall \lambda \in \mathbb{K},v \in F$
\end{itemize}
\subsection{Properties}
Showing that a space is a subspace of a bigger (or equal) vector space if enough to prove that it is itself a vector space.
\section{Relations between Vs and Vss}
\subsection{Linear Combinations}
Let $v_1,v_2,...,v_n$, $n$ vectors from a vector space $E$. Then :\\


\subsection{Vss Intersection}
\subsection{Vss Sum}
\subsection{Generated Vss}
\section{Families}
\subsection{Free Family}
\subsection{Spanning Family}
\section{Basis}
\section{Dimension of a Vss}
\section{Proofs}
\subsection{Intersection of linear subspaces}
\subsection{Sum of linear subspaces}
\subsection{Spanned linear subspaces}
\subsection{Basis}
\subsection{Existence of a basis in a finite dimention}

\end{document}