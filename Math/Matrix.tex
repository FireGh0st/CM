\documentclass{math}
\usepackage{lipsum}
\title{Matrix Chap. 13} %Titre du fichie
\author{FireGhost} %Auteur du fichier

\begin{document}
\titre{Chapter 13: Matrix} %Titre du fichier .pdf
\UE{Matrix} %Nom de la UE


\fairetitre
\tableofcontents
\fairemarges
\newpage
% subsubsubsection
\setcounter{secnumdepth}{4}

\titleformat{\paragraph}
{\normalfont\normalsize\bfseries}{\theparagraph}{1em}{}
\titlespacing*{\paragraph}
{0pt}{3.25ex plus 1ex minus .2ex}{1.5ex plus .2ex}

%
\section{General approach}
\subsection{Definition}
\subsubsection{Definition of a matrix}
We call matrix of n rows and p columns any mapping in the following form:
\[\begin{matrix}
    \llbracket 1,n \rrbracket \times \llbracket 1,p \rrbracket & \rightarrow \mathbb{K} \\
    i,j &  a_{ij} 
\end{matrix}\]

We denote such maps as tables of n rows and p columns, and we write:

\[\begin{pmatrix}
    a_{11} & a_{12} & \cdots & a_{1p} \\
    a_{21} & a_{22} & \cdots & a_{2p} \\
    \vdots & \vdots & \ddots & \vdots \\
    a_{n1} & a_{n2} & \cdots & a_{np}
\end{pmatrix}\]

$\forall(i,j)\in \llbracket 1,n \rrbracket \times \llbracket 1,p \rrbracket$, we call $a_{ij}$ a coefficient of the matrix.
In this case coefficient if i-th row and j-th column.

\subsubsection{Notation}
We denote $M_{np}(\mathbb{K})$ the set of matrix of n rows and p columns with coefficient from $\mathbb{K}$.
\subsubsection{Examples}
%centering example

\[ A = \begin{pmatrix}
    1 & 4  \\
    2 & 5 \\
    3 & 6 
\end{pmatrix}\in M_{32}(\mathbb{R})\]
\[
B = \begin{pmatrix}
    i \\
    1 + i \\
    3
\end{pmatrix}\in M_{31}(\mathbb{C})\]

\subsection{Particular matrices}
Let $A \in M_{np}(\mathbb{K})$ then:
\subsubsection{Null matrix}
\begin{enumerate}
    \item  $\lbrack \forall (i,j) \in \llbracket 1,n \rrbracket \times \llbracket 1,p \rrbracket, a_{ij} = 0 \rbrack \Rightarrow \lbrack A = 0_{np} \rbrack$
    We say A is the null matrix $M_{np}(\mathbb{K})$.
\end{enumerate}
\paragraph{Example}
\[ A'=
\begin{pmatrix}
    0 & 0  \\
    0 & 0  \\
    0 & 0 
\end{pmatrix}\in M_{32}(\mathbb{R})\]

\subsubsection{Column matrix}
\begin{enumerate}
    \setcounter{enumi}{1}
    \item $B \in M_{np}(\mathbb{K})$ and $ p = 1 \Rightarrow $ B is a column matrix of n rows
\end{enumerate}
\paragraph{Example}
\[ B' = \begin{pmatrix}
    1 \\
    2 \\
    3
\end{pmatrix} \in M_{31}(\mathbb{R})\]

\subsubsection{Row matrix}
    \begin{enumerate}
        \setcounter{enumi}{2}
        \item $B \in M_{np}(\mathbb{K})$ and $ n = 1 \Rightarrow $ C is a row matrix of p columns
    \end{enumerate}
    \paragraph{Example}
    \[ C' = \begin{pmatrix}
        1 & 2 & 3
    \end{pmatrix} \in M_{13}(\mathbb{R})\]

\subsubsection{Square matrix}
    We call square matrix any matrix with same number of rows and columns.
    We denote $M_{n}(\mathbb{K})$ the set of square matrix of n rows and columns with coefficient from $\mathbb{K}$.
    \begin{enumerate}
        \setcounter{enumi}{3}
        \item $D \in M_{np}(\mathbb{K})$ and $ n = p \Rightarrow $ D is a square matrix denote $M_{n}(\mathbb{K})$
    \end{enumerate}
    \paragraph{Example}
    \[ D' = \begin{pmatrix}
        1 & 2 & 3 \\
        4 & 5 & 6 \\
        7 & 8 & 9
    \end{pmatrix} \in M_{3}(\mathbb{R})\]

\subsubsection{Diagonal matrix}
    \begin{enumerate}
        \setcounter{enumi}{4}
        \item $\forall E \in M_{n}(\mathbb{R})$, if $\forall (i,j) \in \llbracket 1,n \rrbracket ^2, i \neq j \Rightarrow a_{ij} = 0$ then we say E is a diagonal matrix
    \end{enumerate}
    \paragraph{Example}
    \[ E' = \begin{pmatrix}
        1 & 0 \\
        0 & 2
    \end{pmatrix} \in M_{2}(\mathbb{R})\]
    \[ E'' = \begin{pmatrix}
        1 & 0 & 0 \\
        0 & 2 & 0 \\
        0 & 0 & 3
    \end{pmatrix} \in M_{3}(\mathbb{R})\]

\subsubsection{Triangular matrix}
    \begin{enumerate}
        \setcounter{enumi}{5}
        \item $\forall F \in M_{n}(\mathbb{R})$, if $\forall (i,j) \in \llbracket 1,n \rrbracket ^2, i > j \Rightarrow a_{ij} = 0$ then we say F is a lower triangular matrix
        \item $\forall G \in M_{n}(\mathbb{R})$, if $\forall (i,j) \in \llbracket 1,n \rrbracket ^2, i < j \Rightarrow a_{ij} = 0$ then we say G is a upper triangular matrix
    \end{enumerate}
    \paragraph{Example}
    \[ F' = \begin{pmatrix}
        1 & 0 \\
        2 & 3
    \end{pmatrix} \in M_{2}(\mathbb{R})\]
    \[ G' = \begin{pmatrix}
        1 & 2 \\
        0 & 3
    \end{pmatrix} \in M_{2}(\mathbb{R})\]

\subsection{Transposed matrix}
    \subsubsection{Definition}
        Let $A \in M_{np}(\mathbb{K})$. We call transposed matrix of A (or A transpose) a matrix B from $M_{pn}(\mathbb{K})$ such as:
        \[ \forall (i,j) \in \llbracket 1,n \rrbracket \times \llbracket 1,p \rrbracket, a_{ij} = b_{ji}\]
    \subsubsection{Notation}
    We denote B as ${}^t \! A$
    \subsubsection{Example}
    
    \[ A = \begin{pmatrix}
        1 & 2 & 3 \\
        4 & 5 & 6
    \end{pmatrix} \in M_{23}(\mathbb{R})\]
    \[ {}^t \! A = \begin{pmatrix}
        1 & 4 \\
        2 & 5 \\
        3 & 6
    \end{pmatrix} \in M_{32}(\mathbb{R})\]


\subsection{Symmetric matrix}
    \subsubsection{Symmetric}
        If $ {}^t \! A = A $ then we say A is symmetric
        \paragraph{Example}
        \[ A = \begin{pmatrix}
            1 & 2 & 3 \\
            2 & 4 & 5 \\
            3 & 5 & 6
        \end{pmatrix} = {}^t \! A = 
        \begin{pmatrix}
            1 & 2 & 3 \\
            2 & 4 & 5 \\
            3 & 5 & 6
        \end{pmatrix}
            \in M_{3}(\mathbb{R})\]
    \subsubsection{Anti-Symmetric}
        If $ {}^t \! A = -A $ then we say A is Anti-symmetric
        \paragraph{Example}
        \[ A = \begin{pmatrix}
            0 & -2 & 3 \\
            2 & 0 & -5 \\
            -3 & 5 & 0
        \end{pmatrix} = {}^t \! A = 
        \begin{pmatrix}
            0 & 2 & -3 \\
            -2 & 0 & 5 \\
            3 & -5 & 0
        \end{pmatrix}
            \in M_{3}(\mathbb{R})\]
\end{document}
